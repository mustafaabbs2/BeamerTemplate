%!TeX encoding = UTF-8
%!TeX program = xelatex
\documentclass[notheorems, aspectratio=169]{beamer}
% aspectratio: 1610, 149, 54, 43(default), 32

\usepackage{latexsym}
\usepackage{amsmath,amssymb}
\usepackage{mathtools}
\usepackage{color,xcolor}
\usepackage{graphicx}
\usepackage{textcomp}
\usepackage{graphics}
\usepackage{textcomp}   
\usepackage{amsmath,amssymb}
\usepackage{array}
\usepackage{tabularx}
\usepackage{epstopdf}
\usepackage{picinpar}
\usepackage{verbatim}
\usepackage{subfig}
\usepackage{makeidx}
\usepackage{mathtools} 
\usepackage{pdfpages}
\usepackage{textcomp}
\usepackage{emptypage}
\usepackage[section]{placeins}
\usepackage{listings}
\usepackage{caption}
\usepackage{tikz}
\usepackage{tikz-3dplot}
\usetikzlibrary{patterns}
\usetikzlibrary{shadows,fadings}
\usetikzlibrary{positioning}
\usetikzlibrary{shapes.geometric}
\usetikzlibrary{arrows.meta,arrows}
\usepackage[toc,page]{appendix}
\usepackage{amsthm}
\usepackage{lmodern} % 解决 font warning
% \usepackage[UTF8]{ctex}
\usepackage{animate} % insert gif

\usepackage{lipsum} % To generate test text 
\usepackage{ulem} % 下划线,波浪线

\usepackage{listings} % display code on slides; don't forget [fragile] option after \begin{frame}

\usepackage{tcolorbox}
\usepackage{natbib}
\setcitestyle{authoryear}



% ----------------------------------------------
% tikx
\usepackage{framed}

\usepackage{pgf}
\usetikzlibrary{calc,trees,positioning,arrows,chains,shapes.geometric,%
    decorations.pathreplacing,decorations.pathmorphing,shapes,%
    matrix,shapes.symbols}
\pgfmathsetseed{1} % To have predictable results
% Define a background layer, in which the parchment shape is drawn
\pgfdeclarelayer{background}
\pgfsetlayers{background,main}

\definecolor{AmethystPurple}{HTML}{AEAEDF}
\definecolor{myblue}{RGB}{1,75,182}
% define styles for the normal border and the torn border
\tikzset{
  normal border/.style={black, decorate, 
     decoration={random steps, segment length=2.5cm, amplitude=.7mm}},
  torn border/.style={black, decorate, 
     decoration={random steps, segment length=.5cm, amplitude=1.7mm}}}

% Macro to draw the shape behind the text, when it fits completly in the
% page
\def\parchmentframe#1{
\tikz{
  \node[inner sep=1.5em] (A) {#1};  % Draw the text of the node
  \begin{pgfonlayer}{background}  % Draw the shape behind
  \fill[normal border] 
        (A.south east) -- (A.south west) -- 
        (A.north west) -- (A.north east) -- cycle;
  \end{pgfonlayer}}}

% Macro to draw the shape, when the text will continue in next page
\def\parchmentframetop#1{
\tikz{
  \node[inner sep=2em] (A) {#1};    % Draw the text of the node
  \begin{pgfonlayer}{background}    
  \fill[normal border]              % Draw the ``complete shape'' behind
        (A.south east) -- (A.south west) -- 
        (A.north west) -- (A.north east) -- cycle;
  \fill[torn border]                % Add the torn lower border
        ($(A.south east)-(0,.2)$) -- ($(A.south west)-(0,.2)$) -- 
        ($(A.south west)+(0,.2)$) -- ($(A.south east)+(0,.2)$) -- cycle;
  \end{pgfonlayer}}}

% Macro to draw the shape, when the text continues from previous page
\def\parchmentframebottom#1{
\tikz{
  \node[inner sep=2em] (A) {#1};   % Draw the text of the node
  \begin{pgfonlayer}{background}   
  \fill[normal border]             % Draw the ``complete shape'' behind
        (A.south east) -- (A.south west) -- 
        (A.north west) -- (A.north east) -- cycle;
  \fill[torn border]               % Add the torn upper border
        ($(A.north east)-(0,.2)$) -- ($(A.north west)-(0,.2)$) -- 
        ($(A.north west)+(0,.2)$) -- ($(A.north east)+(0,.2)$) -- cycle;
  \end{pgfonlayer}}}

% Macro to draw the shape, when both the text continues from previous page
% and it will continue in next page
\def\parchmentframemiddle#1{
\tikz{
  \node[inner sep=2em] (A) {#1};   % Draw the text of the node
  \begin{pgfonlayer}{background}   
  \fill[normal border]             % Draw the ``complete shape'' behind
        (A.south east) -- (A.south west) -- 
        (A.north west) -- (A.north east) -- cycle;
  \fill[torn border]               % Add the torn lower border
        ($(A.south east)-(0,.2)$) -- ($(A.south west)-(0,.2)$) -- 
        ($(A.south west)+(0,.2)$) -- ($(A.south east)+(0,.2)$) -- cycle;
  \fill[torn border]               % Add the torn upper border
        ($(A.north east)-(0,.2)$) -- ($(A.north west)-(0,.2)$) -- 
        ($(A.north west)+(0,.2)$) -- ($(A.north east)+(0,.2)$) -- cycle;
  \end{pgfonlayer}}}

% Define the environment which puts the frame
% In this case, the environment also accepts an argument with an optional
% title (which defaults to ``Example'', which is typeset in a box overlaid
% on the top border
\newenvironment{parchment}[1][Example]{%
  \def\FrameCommand{\parchmentframe}%
  \def\FirstFrameCommand{\parchmentframetop}%
  \def\LastFrameCommand{\parchmentframebottom}%
  \def\MidFrameCommand{\parchmentframemiddle}%
  \vskip\baselineskip
  \MakeFramed {\FrameRestore}
  \noindent\tikz\node[inner sep=1ex, draw=black!20,fill=black, 
          anchor=west, overlay] at (0em, 1em) {\sffamily#1};\par}%
{\endMakeFramed}

% ----------------------------------------------

\mode<presentation>{
    \usetheme{Warsaw}
    % Boadilla CambridgeUS
    % default Antibes Berlin Copenhagen
    % Madrid Montpelier Ilmenau Malmoe
    % Berkeley Singapore Warsaw
    \usecolortheme{orchid}
    % beetle, beaver, orchid, whale, dolphin
    \useoutertheme{infolines}
    % infolines miniframes shadow sidebar smoothbars smoothtree split tree
    \useinnertheme{circles}
    % circles, rectanges, rounded, inmargin
}
% 设置 block 颜色
\setbeamercolor{block title}{bg=black,fg=white}

\newcommand{\reditem}[1]{\setbeamercolor{item}{fg=red}\item #1}

% 缩放公式大小
\newcommand*{\Scale}[2][4]{\scalebox{#1}{\ensuremath{#2}}}

% 解决 font warning
\renewcommand\textbullet{\ensuremath{\bullet}}

% ---------------------------------------------------------------------
% flow chart
\tikzset{
    >=stealth',
    punktchain/.style={
        rectangle, 
        rounded corners, 
        % fill=black!10,
        draw=white, very thick,
        text width=6em,
        minimum height=2em, 
        text centered, 
        on chain
    },
    largepunktchain/.style={
        rectangle,
        rounded corners,
        draw=white, very thick,
        text width=10em,
        minimum height=2em,
        on chain
    },
    line/.style={draw, thick, <-},
    element/.style={
        tape,
        top color=white,
        bottom color=blue!50!black!60!,
        minimum width=6em,
        draw=blue!40!black!90, very thick,
        text width=6em, 
        minimum height=2em, 
        text centered, 
        on chain
    },
    every join/.style={->, thick,shorten >=1pt},
    decoration={brace},
    tuborg/.style={decorate},
    tubnode/.style={midway, right=2pt},
    font={\fontsize{10pt}{12}\selectfont},
}
% ---------------------------------------------------------------------

% code setting
\lstset{
    language=C++,
    basicstyle=\ttfamily\footnotesize,
    keywordstyle=\color{red},
    breaklines=true,
    xleftmargin=2em,
    frame=shadowbox,
    rulecolor=\color{black},
    commentstyle=\color{lightgray}
    tabsize=4,
    breakatwhitespace=false,
    showspaces=false,               
    showstringspaces=false,
    showtabs=false,
    morekeywords={Str, Num, List},
}

% ---------------------------------------------------------------------

%% preamble
\title[Diving into SPH]{Diving into SPH}
\institute[]{}
\date{}
% -------------------------------------------------------------



\begin{document}

%% title frame
\begin{frame}
    \titlepage
\end{frame}


\section{SPH: What I've understood} 
\frame{\tableofcontents[currentsection]}

\subsection{Recap}

\begin{frame}[fragile]
  \frametitle{Navier-Stokes Equations}
  \framesubtitle{(Still N-S!)}
  
  The key idea is unchanged. You want to approximate the N-S equations as before, but not discretize your solution on a mesh. This is like any Lagrangian method. To recap the equations for continuum flow:
  
  \begin{itemize}
    \item Continuity Equation:
  \end{itemize}
  
  \[
  \frac{D\rho}{Dt} = -\rho \nabla \cdot \mathbf{u}
  \]
  
  \begin{itemize}
    \item Momentum Equation:
  \end{itemize}
  
  \[
  \frac{D\mathbf{u}}{Dt} = -\frac{1}{\rho}\nabla P + \nu\nabla^2\mathbf{u} + \mathbf{f}
  \]
  
  
  \end{frame}

  
\begin{frame}[fragile]
  \frametitle{Smoothed Particle Hydrodynamics (SPH)}
  \framesubtitle{What makes it different}
  
  \begin{itemize}
    \item SPH is mesh-free!
    \item Free surface flows that are challenging in Eulerian methods are more or less natural in SPH
  \end{itemize}
  
  \end{frame}
  

  \subsection{Equations}

  \begin{frame}
    \frametitle{Kernels}
    \framesubtitle{Mathematical Form}

    In 3D (also what I've implemented)
    \begin{equation}
      W(q) = \frac{21}{64\pi h^3} \cdot
      \begin{dcases}
        (2 - q)^4 (1 + 2q) & \text{if } 0 \leq q \leq 2 \\
        0 & \text{otherwise}
      \end{dcases}
      \end{equation}
    
    where:
    \begin{itemize}
      \item Kernels just convolve over the function, and aim to be an approximation to the Dirac Delta function
      \item It has compact support, which means the tail is not infitine, it takes a certain range of particles around it
    \end{itemize}
    
    \end{frame}
    



  \begin{frame}[fragile]
  \frametitle{SPH Equations}
  \framesubtitle{Continuity}
  
  The approach in SPH is sto sum up contributions from neighbours.
  
  \[
  \rho_a = \sum_{b} m_b \frac{W(\mathbf{r}_{ab}, h)}{\rho_b}
  \]
  
  % \begin{lstlisting}
  % rho[i] = sum_j(m[j] * W(r_ij, h) / rho[j])
  % \end{lstlisting}
  
  where:
  \begin{itemize}
    \item \( W \) is the smoothing kernel function.
    \item \( \mathbf{r}_{ab} \) is the vector from particle \( a \) to \( b \).
    \item \( h \) is the smoothing length.
  \end{itemize}
  
  \end{frame}



    \begin{frame}[fragile]
      \frametitle{SPH Equations}
      \framesubtitle{Momentum}
      
      The kernel functions can be differentiated - and this is used in the gradient for pressure in the momentum equation. Pressure is calculated from the equation of state. 
      
      \begin{equation*}
      \frac{D\mathbf{v}_a}{Dt} = -\sum_b m_b \left( \frac{P_a}{\rho_a^2} + \frac{P_b}{\rho_b^2} \right) \nabla_a W_{ab}(h)
      + \nu \nabla^2 \mathbf{v}_a + \mathbf{f}_a
      \end{equation*}
      
      where:
      \begin{itemize}

        \item \( P_a \) and \( P_b \), \( \rho_a \) and \( \rho_b \) are the pressures and densities of the current particle and its neighbours 
        \item \( \nabla_a W_{ab}(h) \) is the gradient of the smoothing kernel between particles \( a \) and \( b \) with smoothing length \( h \).
      \end{itemize}
      
      \end{frame}

      
      \section{Postprocessing} 
      \frame{\tableofcontents[currentsection]}

  \section{What I tried} 
  \frame{\tableofcontents[currentsection]}
      




\end{document}